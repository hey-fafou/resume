\documentclass[10pt,a4paper]{moderncv}
\moderncvtheme[black]{classic}  
\usepackage[top=0.8cm, bottom=0.8cm, left=1.1cm, right=1.8cm]{geometry}
\usepackage[frenchb]{babel}
\usepackage[utf8]{inputenc}

% Largeur de la colonne pour les dates
\setlength{\hintscolumnwidth}{3.4cm}


\firstname{Wilfred}
\familyname{Varma}     
\title{Ingénieur en robotique}
\address{6 rond point de Picardie}{77176 Savigny-le-Temple}    
\email{wilfred.varma@gmail.com}                      
\mobile{06 60 79 82 50} 
\extrainfo{25 ans -- Permis B, véhiculé}
\photo[64pt][0pt]{img/Profil.jpg}
\begin{document}
\rmfamily
\maketitle

\section{Expériences professionnelles}

\cventry{Septembre 2015\\à aujourd'hui}{Ingénieur conception et développement informatique}{NovaSparks}{Paris}{France}{\begin{itemize}%
\item Développement C++ Objet pour application à la \textit{market data}
\item Architecture Logicielle et Design Pattern
\item Multithreading
\item Programmation réseau TCP/IP, UDP/IP\newline{}
\end{itemize}}

\cventry{Fevrier 2015\\à Juillet 2015}{Stage 6 mois -- Implémentation et évaluation de ROS sur chariots automatiques}{Balyo}{Moissy-Cramayel (77)}{France}{\begin{itemize}%
\item Conception d'une architecture modulaire
\item Développement d'un \textit{driver} pour communication avec la GPIO du chariot
\item Développement des différentes couches de sécurités
\item Développement des asservissements de vitesse, d'angle, et position des fourches
\item Localisation du chariot avec filtre particulaire et algorithme de SLAM
\item Navigation du chariot sur un circuit prédéfinit
\item Développement d'une machine a état dans un noeud ROS python\newline{}
\end{itemize}}

\cventry{Juin 2014\\à Août 2014}{Stage 3 mois -- Validation QSE d'une ligne de fabrication d'e-liquide}{Laboratoires Phodé}{Terssac (81)}{France}{\begin{itemize}%
\item Formalisation du processus de fabrication
\item Tests de validation du processus
\item Lancement de la production
\item Conception d'un dossier QSE pour la ligne de fabrication\newline{}
\end{itemize}}

\cventry{Juin 2013\\à  Août 2013}{Stage 3 mois -- développement .Net d'une extension Microsoft Outlook}{Integra Software Services}{Pondicherry}{Inde}{
\begin{itemize}
\item Rédaction du cahier des charges
\item Développement informatique
\item Tests d'intégration et validation\newline{}
\end{itemize}}

\section{Formations}
\cventry{2012 -- 2015}{Diplôme d'ingéneur}{Polytech Paris Pierre et Marie Curie}{}{}{Spécialité Robotique}
\cventry{2009 -- 2012}{Licence Mathématique Informatique et Applications}{Université Paris Ouest la Défense}{Option électronique}{}{}

\section{Compétences techniques}
\cvitem{Développement}{C, C++, bash, python}
\cvitem{Logiciels}{Matlab, SolidWorks, ROS}
\cvitem{Versionnage}{Git, SVN}
\cvlanguage{Anglais}{Courant -- Toeic 910 / 990}{}

\section{Centres d'intérêts}
\cvitem{Sport}{14 années de pratiques de \textbf{Karaté}, ceinture noire 3\ieme{} dan, compétiteur national}
\cvitem{Associatif}{
\textbf{Président de PolyChallenge 2013-2014} à Polytech Paris UPMC :
\begin{itemize}
 \item Organisation d'un marathon d'innovation sur 24 heures.\newline{}
\end{itemize}
\textbf{Webmaster} du club de Karaté ASPS Plessis Savigny :
\begin{itemize}
 \item Conception d'un site web: \emph{http://savigny-kata.fr}
\end{itemize}
}
\end{document}
